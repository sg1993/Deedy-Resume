%%%%%%%%%%%%%%%%%%%%%%%%%%%%%%%%%%%%%%%
% Deedy - One Page Two Column Resume
% LaTeX Template
% Version 1.2 (16/9/2014)
%
% Original author:
% Debarghya Das (http://debarghyadas.com)
%
% Original repository:
% https://github.com/deedydas/Deedy-Resume
%
% IMPORTANT: THIS TEMPLATE NEEDS TO BE COMPILED WITH XeLaTeX
%
% This template uses several fonts not included with Windows/Linux by
% default. If you get compilation errors saying a font is missing, find the line
% on which the font is used and either change it to a font included with your
% operating system or comment the line out to use the default font.
% 
%%%%%%%%%%%%%%%%%%%%%%%%%%%%%%%%%%%%%%
% 
% TODO:
% 1. Integrate biber/bibtex for article citation under publications.
% 2. Figure out a smoother way for the document to flow onto the next page.
% 3. Add styling information for a "Projects/Hacks" section.
% 4. Add location/address information
% 5. Merge OpenFont and MacFonts as a single sty with options.
% 
%%%%%%%%%%%%%%%%%%%%%%%%%%%%%%%%%%%%%%
%
% CHANGELOG:
% v1.1:
% 1. Fixed several compilation bugs with \renewcommand
% 2. Got Open-source fonts (Windows/Linux support)
% 3. Added Last Updated
% 4. Move Title styling into .sty
% 5. Commented .sty file.
%
%%%%%%%%%%%%%%%%%%%%%%%%%%%%%%%%%%%%%%%
%
% Known Issues:
% 1. Overflows onto second page if any column's contents are more than the
% vertical limit
% 2. Hacky space on the first bullet point on the second column.
%
%%%%%%%%%%%%%%%%%%%%%%%%%%%%%%%%%%%%%%


\documentclass[]{deedy-resume-openfont}
\usepackage{fancyhdr}
\usepackage{fontawesome}
\usepackage{url}
 
\pagestyle{fancy}
\fancyhf{}
 
\begin{document}

%%%%%%%%%%%%%%%%%%%%%%%%%%%%%%%%%%%%%%
%
%     LAST UPDATED DATE
%
%%%%%%%%%%%%%%%%%%%%%%%%%%%%%%%%%%%%%%
\lastupdated

%%%%%%%%%%%%%%%%%%%%%%%%%%%%%%%%%%%%%%
%
%     TITLE NAME
%
%%%%%%%%%%%%%%%%%%%%%%%%%%%%%%%%%%%%%%
\namesection{Shibin}{George}{ \urlstyle{same}\href{https://www.linkedin.com/in/sg1993}{\faLinkedinSquare } | \href{https://github.com/sg1993}{\faGithubSquare }\\
\href{mailto:shibingeorge@cs.umass.edu}{\faEnvelope \space shibingeorge@cs.umass.edu} | {\faMobile \space (+1) 413.539.8032}
}

%%%%%%%%%%%%%%%%%%%%%%%%%%%%%%%%%%%%%%
%
%     COLUMN ONE
%
%%%%%%%%%%%%%%%%%%%%%%%%%%%%%%%%%%%%%%

\begin{minipage}[t]{0.33\textwidth} 

%%%%%%%%%%%%%%%%%%%%%%%%%%%%%%%%%%%%%%
%     EDUCATION
%%%%%%%%%%%%%%%%%%%%%%%%%%%%%%%%%%%%%%

\section{Education} 

\subsection{UMass Amherst}
\location {Amherst, MA}
\descript{M.S. in Computer Science}
\location{Sep, 2019 - Dec, 2020 (Expected)}
\sectionsep

\subsection{NIT Warangal}
\location {Warangal, India}
\descript{B.Tech in Computer Science}
\location{August, 2011 - April, 2015 }
\location{ GPA: 8.48 / 10.0 }
\sectionsep

%%%%%%%%%%%%%%%%%%%%%%%%%%%%%%%%%%%%%%
%     LINKS
%%%%%%%%%%%%%%%%%%%%%%%%%%%%%%%%%%%%%%

%\section{Links} 
%Facebook:// \href{https://facebook/dd}{\bf dd} \\
%Github:// \href{https://github.com/deedydas}{\bf deedydas} \\
%LinkedIn://  \href{https://www.linkedin.com/in/debarghyadas}{\bf debarghyadas} \\
%YouTube://  \href{https://www.youtube.com/user/DeedyDash007}{\bf DeedyDash007} \\
%Twitter://  \href{https://twitter.com/debarghya_das}{\bf @debarghya\_das} \\
%Quora://  \href{https://www.quora.com/Debarghya-Das}{\bf Debarghya-Das}

%%%%%%%%%%%%%%%%%%%%%%%%%%%%%%%%%%%%%%
%     COURSEWORK
%%%%%%%%%%%%%%%%%%%%%%%%%%%%%%%%%%%%%%

\section{Coursework}
\subsection{Graduate}
Information Retrieval \\
Natural Language Processing \\
Neural Networks \\
\sectionsep

\subsection{Undergraduate}
Operating Systems + Practicum \\
Distributed Computing  \\

%%%%%%%%%%%%%%%%%%%%%%%%%%%%%%%%%%%%%%
%     SKILLS
%%%%%%%%%%%%%%%%%%%%%%%%%%%%%%%%%%%%%%

\section{Skills}
\subsection{Programming}
\location{Over 5000 lines:}
Java \textbullet{} C \textbullet{} C++ \textbullet{} Shell \textbullet{} Python \\
\location{Over 1000 lines:}
Matlab \textbullet{} Javascript \\
\sectionsep

%%%%%%%%%%%%%%%%%%%%%%%%%%%%%%%%%%%%%%
%     PUBLICATIONS
%%%%%%%%%%%%%%%%%%%%%%%%%%%%%%%%%%%%%%

\section{Publications} 
\renewcommand\refname{\vskip -0.75cm} % Couldn't get this working from the .cls file
\Urlmuskip=0mu plus 1mu\relax
\bibliographystyle{abbrv}
\bibliography{publications}
\nocite{*}

%%%%%%%%%%%%%%%%%%%%%%%%%%%%%%%%%%%%%%
%
%     COLUMN TWO
%
%%%%%%%%%%%%%%%%%%%%%%%%%%%%%%%%%%%%%%

\end{minipage} 
\hfill
\begin{minipage}[t]{0.66\textwidth} 

%%%%%%%%%%%%%%%%%%%%%%%%%%%%%%%%%%%%%%
%     EXPERIENCE
%%%%%%%%%%%%%%%%%%%%%%%%%%%%%%%%%%%%%%

\section{Experience}
\runsubsection{Qualcomm}
\descript{| Senior Software Engineer }
\location{Jun 2015 - July 2019 | Hyderabad, India}
\vspace{\topsep} % Hacky fix for awkward extra vertical space
\begin{tightemize}
\item Contributed bug-fixes to the Android Open Source Project, the official repository that hosts the Android OS. Take a look at my contributions \textbf{\href{https://android-review.googlesource.com/q/shibing}{here}}.
\item Designed and implemented Over-The-Air (OTA) upgrade solutions for Linux Android platform. I was awarded a \textbf{\href{https://www.slideshare.net/slideshow/embed_code/key/uhvsweQaLamkiG} {Super Qualstar}} (highest recognition for individual-contribution within Qualcomm) for my work on A/B OTA upgrade solution.
\item Developing/maintaining \textbf{\href{https://source.codeaurora.org/quic/la/platform/bionic/commit/?h=android-framework.lnx.3.1&id=0c0fee17b7f39e841f6ef5d305adb1d5189dfb25}{tools}} that facilitate debugging (of memory (heap)-leaks and heap-corruptions)
\end{tightemize}
\sectionsep

\runsubsection{Qualcomm}
\descript{| Software Engineering Intern }
\location{May 2014 - July 2014 | Hyderabad, India}
\vspace{\topsep} % Hacky fix for awkward extra vertical space
\begin{tightemize}
\item Worked on an optimization designed for the Android composition engine (SurfaceFlinger) and built a prototype for the same.
\end{tightemize}
\sectionsep

%%%%%%%%%%%%%%%%%%%%%%%%%%%%%%%%%%%%%%
%     RESEARCH
%%%%%%%%%%%%%%%%%%%%%%%%%%%%%%%%%%%%%%

\section{Projects}
\runsubsection{CBIR using Local-Tetra Patterns on Hadoop MapReduce framework}
\descript{}
Content-based Image retrieval (CBIR) involves extracting features (Local Tetra Patterns in this case) from every image in a dataset and then retrieving images from the dataset that is closest to a user-specified query-image.
There were 3 stages involved: taking a huge dataset of images and converting them to MapReduce's native SequenceFile type (\textbf{\href{https://github.com/sg1993/sequencify-CBIR-on-hadoop}{github-stage-1}}), extracting image features from SequenceFiles and storing the features on Hadoop's Distributed FileSystem (\textbf{\href{https://github.com/sg1993/CBIR-on-Hadoop}{github-stage-2}}), and then fetching the results to a user-specified query(\textbf{\href{https://github.com/sg1993/CBIR-query-on-Hadoop}{github-stage-3}}). Also wrote a \textbf{\href{https://github.com/sg1993/CBIR-query-Image-Viewer}{image-viewer GUI}} using Swing to view the results from a CBIR query. See [1] in Publications.
\sectionsep

\runsubsection{Weighted finite automata encoding of images}
\descript{}
The idea was to explore how feature extraction from an image can be done using Weighted finite automata (WFA) encoding of the same image. WFA encoding is a technique primarily meant for image compression but my focus was on exploring its applicability on tamper-detection in images. See [2] in Publications.
\sectionsep

\runsubsection{Save The Session}
\descript{}
\textbf{\href{https://chrome.google.com/webstore/detail/save-the-session/gfokkgedgncpmhnbomipnbnpkedjpbil}{Save The Session}} is a Chrome extension for saving user sessions and reloading them later. It has about 4000 active users \enspace - \enspace 
{\textbf{\href{https://github.com/sg1993/Save-The-Session}{Github}}}
\sectionsep

%%%%%%%%%%%%%%%%%%%%%%%%%%%%%%%%%%%%%%
%     AWARDS
%%%%%%%%%%%%%%%%%%%%%%%%%%%%%%%%%%%%%%

\section{Awards} 
\begin{tabular}{rll}
2014	     & top 52/2500  & KPCB Engineering Fellow\\
2014	     & 1\textsuperscript{st}/50  & Microsoft Coding Competition, Cornell\\
2013	     & National  & Jump Trading Challenge Finalist\\
2013     & 7\textsuperscript{th}/120 & CS 3410 Cache Race Bot Tournament  \\
2012     & 2\textsuperscript{nd}/150 & CS 3110 Biannual Intra-Class Bot Tournament \\
2011     & National & Indian National Mathematics Olympiad (INMO) Finalist \\
\end{tabular}
\sectionsep

\end{minipage} 
\end{document}  \documentclass[]{article}
